\documentclass[11pt, a4paper]{article}

\usepackage[utf8]{inputenc}
\usepackage[a4paper, total={6in, 8in}]{geometry}
\raggedbottom

\usepackage{mlmodern}
\usepackage{csquotes}
\usepackage{parskip}
\usepackage[italian]{babel}
\usepackage{booktabs}
\usepackage{multicol}
\usepackage{graphicx}
\usepackage{fancyhdr}
\usepackage{float}
\usepackage{changepage}
\usepackage{fancyvrb}
\usepackage[final]{pdfpages}
\usepackage{placeins}
\usepackage{hyperref}
\usepackage[all]{hypcap}
\usepackage{tcolorbox}
\usepackage{xurl}
\usepackage{xcolor}
\usepackage{tikz}
\usepackage{biblatex}
\usepackage{array}
\usepackage{pdfpages}
\usepackage{subfig}
\usepackage{multirow}
\usepackage{pgfplots}
\usepackage{listings}
\usepgfplotslibrary{groupplots}
\pgfplotsset{compat=1.18}


\addbibresource{bibliography.bib}

\hypersetup{
	colorlinks=false,
	citecolor=black,
	filecolor=black,
	linkcolor=black,
	urlcolor=black,
	allbordercolors=white
}


\definecolor{codegreen}{rgb}{0,0.6,0}
\definecolor{codegray}{rgb}{0.5,0.5,0.5}
\definecolor{codepurple}{rgb}{0.58,0,0.82}
\definecolor{backcolour}{rgb}{0.95,0.95,0.92}

\lstdefinestyle{mystyle}{
	backgroundcolor=\color{backcolour},   
	commentstyle=\color{codegreen},
	numberstyle=\tiny\color{codegray},
	stringstyle=\color{codepurple},
	basicstyle=\ttfamily\footnotesize,
	breakatwhitespace=false,         
	breaklines=true,                 
	captionpos=b,                    
	keepspaces=true,                   
	numbersep=5pt,                  
	showspaces=false,                
	showstringspaces=false,
	showtabs=true,                  
	tabsize=1
}

\lstset{style=mystyle}

\setlength{\headheight}{14pt}

\begin{document}
	\begin{center}
	{\LARGE \textbf{Università degli Studi di Milano Bicocca}} \\
	\vspace{0.2cm}
	{\Large {Dipartimento di Informatica}} \\ 
	\vspace{1cm}
	
	\includegraphics[width=4cm]{figures/unimib-logo.png} \\
	\vspace{0.4cm}
	
	{\huge \textbf{Metodi del calcolo scientifico}} \\ 
	\Large{Libreria MatrixDestoryer per la risoluzione di sistemi lineari in modo iterativi}
	\vspace{1cm}
\end{center}
\vspace{2cm}

\noindent \large{Membri del gruppo:} 

\begin{tabular}{@{}lll}
	\textbf{Canesi Gabriele} & 851637 & \href{mailto:g.canesi1@campus.unimib.it}{\texttt{g.canesi1@campus.unimib.it}}\\
	\textbf{Fiorenza Gioele} & 851631 & \href{mailto:g.fiorenza1@campus.unimib.it}{\texttt{g.fiorenza1@campus.unimib.it}}\\
	\textbf{Ghislotti Gianluca} & 859242 & \href{mailto:g.ghislotti@campus.unimib.it}{\texttt{g.ghislotti@campus.unimib.it}}\\
\end{tabular}

\vfill
\begin{center}
	\large{\textbf{Anno Accademico 2023/2024}}
\end{center}




	\thispagestyle{empty}
	\newpage
	\pagestyle{fancy}
	\pagenumbering{roman}
	
	
	\tableofcontents
	
	\newpage
	
	\definecolor{fireenginered}{rgb}{0.81, 0.09, 0.13}
	\hypersetup{
		linkbordercolor=fireenginered
	}
	\pagenumbering{arabic}
	
	\section{Introduzione}

Risolvere sistemi lineari con matrici di elevate dimensioni può rappresentare un problema per tutti i metodi di risoluzione, in quanto la memoria richiesta potrebbe non essere sufficiente per mantenere tutto il contenuto della matrice.

Solitamente, però, le matrici di queste dimensioni sono sparse, ovvero contengono poche entrate diverse da 0. Tale informazione consente di definire un insieme di metodi definiti \textbf{iterativi}, i quali basandosi sulla matrice iniziale e sulla soluzione corrente ne generano una successiva. La loro continua applicazione permette di ridurre man mano l'errore rispetto alla soluzione corretta.

Si potrebbe pensare di applicare i metodi \textbf{diretti} per generare la soluzione corretta, ma questi, se le matrici sono state salvate tramite la rappresentazione sparsa, sono soggetti al fenomeno del \textbf{fill-in}, ossia durante l'esecuzione dell'algoritmo vengono generate delle matrici intermedie piene. Ciò comporta la perdita dei vantaggi forniti dalle matrici sparse, implicando anche un maggiore utilizzo della memoria del calcolare. Nei casi in cui la matrice sparsa sia di elevate dimensioni, la memoria richiesta potrebbe superare anche quella disponibile su un normale calcolatore.
 
Nel seguente progetto viene proposta una libreria, basata su Eigen \cite{Eigen} , il cui scopo è quello di risolvere iterativamente dei sistemi lineari. Attraverso l'interfaccia grafica realizzata si possono osservare i dati in dei grafici, oltre che visualizzare i risultati in una tabella. Nonostante il focus del progetto riguardasse il confronto dei metodi attraverso lo script più "esterno", la libreria vera e propria consente di risolvere qualunque sistema lineare in modo iterativo, fornendo i parametri relativi alla matrice dei coefficienti \textbf{A} e il vettore dei termini noti \textbf{b}. Inoltre si possono specificare alcune informazioni aggiuntive, come il metodo da utilizzare o alcuni parametri specifici per le versioni rilassate dei metodi che li supportano.
 
 
 
	\section{Architettura}

Il nostro progetto contiene tre progetti distinti:

\begin{enumerate}
	\item La libreria effettiva (cartella \path{lib/})
	\item Una piccola applicazione a riga di comando (cartella \path{cli/})
	\item Un'applicazione a interfaccia grafica (cartella \path{QTInterface/})
\end{enumerate}

La libreria viene compilata producendo in output un file binario non eseguibile; per poterla utilizzare, gli altri due sottoprogetti (indipendenti tra loro) vengono linkati con esso per poter produrre un file eseguibile. Questa struttura permette di scindere in modo netto il codice dei nostri applicativi "client" da quello della libreria, incentivando al riutilizzo in contesti diversi.

\subsection{Libreria effettiva}
Come citato in precedenza, questo è il sottoprogetto principale tra quelli presenti. In Figura \ref{fig:libdiagram} si può osservare l'architettura delle classi di questo progetto, con l'omissione di alcuni tipi nelle firme dei metodi e il caricamento in formato vettoriale per questioni di leggibilità. Il componente più importante è la classe \texttt{IterativeSolver}. Questa classe estende (e implementa, essendo concreta) la classe astratta \texttt{Solver}. All'interno di queste due classi  è presente un metodo \texttt{solve} che prende in input la matrice dei coefficienti, il vettore dei termini noti e restituisce il vettore soluzione calcolato. Il punto fondamentale della risoluzione di un sistema in modo iterativo è il ciclo che esegue gli update a partire dalla soluzione corrente: essendo uguale per qualsiasi metodo, abbiamo fatto in modo che il solver iterativo avesse come attributo di istanza un puntatore alla regola astratta di update,  che viene estesa e implementata dalle regole specifiche, istanziate da chi usa la libreria, sulla base delle proprie esigenze. 

\begin{lstlisting}[caption={Loop del solver}, label={lst:update}, float]
	updateStrategy->init(A, b);
	
	do {
		currentResult = updateStrategy->update();
		updateResidual(*currentResult, A, b);
		++iter;
	} while (iter < maxIter && !reachedTolerance(b, tol));
\end{lstlisting}

Il punto fondamentale della risoluzione di un sistema in modo iterativo è il ciclo che esegue gli update, osservabile nel Listato \ref{lst:update}. Nel suo corpo è presente un riferimento al metodo di update astratto, chiamato \textit{updateStrategy}, sul quale viene prima invocato un primo metodo, init, che esegue alcune operazioni preliminari legate al metodo, come ad esempio la matrice $P^-1$ per Jacobi. Successivamente, nella generazione delle varie soluzioni, viene chiamato il metodo di update, sulla strategia selezionata. Infine viene calcolato il residuo, in modo tale che venga computato una sola volta e condiviso tra il metodo di arresto e la strategia concreta.

Essendo \textit{updateStrategy} astratto, un utente che utilizza la libreria può specificare il metodo che preferisce tra quelli implementati, oppure può anche crearsi una propria implementazione da passare poi al solver. Questo lascia anche aperta la possibilità di implementare facilmente altri metodi iterativi. Questa struttura è un'applicazione del design pattern \textit{Strategy} \cite{Strategy}. Esiste una classe concreta per ognuno dei quattro metodi richiesti.




Questa libreria, inoltre, permette di creare dei solver personalizzando il tipo di matrice dei coefficienti (sparsa o densa) e il tipo di precisione (\path{float} o \path{double}) utilizzando in modo esteso i template di C++.

\begin{figure}
	\centering
	\includegraphics[width=\textwidth]{figures/libDiagram.pdf}
	\caption{Diagramma delle classi della libreria principale}
	\label{fig:libdiagram}
\end{figure}

\subsection{Applicazione da riga di comando} \label{sec:cli}
Questo sottoprogetto è molto piccolo, di fatto contiene una cli che accetta due sotto-comandi passati come stringhe:

\begin{enumerate}
	\item \path{solve}, che permette di specificare il percorso relativo ad una matrice da risolvere, applicandone i test creati.
	\item \path{test}, attraverso il quale è possibile eseguire i test su tutte le varie matrici di benchmark. È importante notare che, nel caso di test, le matrici devono trovarsi nella cartella \path{../Matrices/}.
\end{enumerate}

È importante notare che, a entrambi i comandi, si possono passare dei parametri utili a modificare il comportamento del solver. Nello specifico, sono presenti i seguenti:

\begin{itemize}
	\item \path{--jacobiW} indica il parametro $\omega$ per il metodo di Jacobi;
	\item \path{--gaussW} si riferisce al parametro $\omega$ per Gauss-Seidel;
	\item \path{--skipMatrixCheck} è utile per specificare se non serve controllare che la matrice data in input sia simmetrica e definita positiva;
	\item \path{--norm} permette di definire che tipo di norma usare nel controllo di convergenza. Le opzioni possibili sono \path{eucledian}, \path{manhattan} e \path{infinity}.
\end{itemize}

Per mantenere una coerenza tra i sottoprogetti, i test richiamano un header comune, il \path{runner.h}, il quale esegue le chiamate alla libreria sottostante e ne riporta i benchmark. Essendo in comune tra i progetti, esso è stato posto nella cartella principale, piuttosto che nella sottocartella \path{cli/}.


\subsection{Applicazione GUI}
Abbiamo creato anche una piccola applicazione con interfaccia grafica usando il framework Qt \cite{Qt}. Questa permette di:
\begin{itemize}
	\item Selezionare il file contenente la matrice de coefficienti
	\item Regolare i parametri $\omega$ per i metodi di Jacobi e Gauss-Seidel;
	\item Selezionare il tipo di norma da calcolare in fase di controllo della convergenza. Le scelte possibili sono euclidea, manhattan e infinito;
	\item Decidere di saltare il controllo che stabilisce se la matrice è simmetrica e definita positiva\footnote{Questa opzione può tornare utile per eventuali benchmark. Infatti, questo controllo richiede diverso tempo e rischia di nascondere le differenze tra i vari metodi in termini di tempo, soprattutto per matrici relativamente piccole};
	\item Visualizzare i risultati dei metodi sotto forma di tabella e di grafici che mostrano l'errore relativo e i tempi in funzione della tolleranza, raggruppati per metodo (Figura \ref{fig:ui:output}).
	\item Esportare i risultati in formato CSV.
\end{itemize}

Oltre a queste funzionalità, l'applicazione GUI mostra a schermo una finestra di avviso nel caso in cui l'inverso del  numero di condizionamento stimato sia troppo vicino a 0, al fine di informare l'utente di possibili errori molto alti nella risoluzione del sistema.

\begin{figure}%
	\centering
	\includegraphics[width=0.5\textwidth]{figures/UI/main.png}
	\caption{Schermata iniziale dell'applicazione}%
	\label{fig:ui:main}%
\end{figure}

\begin{figure}%
	\centering
	\subfloat[\centering  Tabella riassuntiva]{{\includegraphics[width=0.45\textwidth]{figures/UI/table} }}%
	\qquad
	\subfloat[\centering Grafico tolleranza / tempi]{{\includegraphics[width=0.45\textwidth]{figures/UI/chart} }}%
	\caption{Esempi di output prodotti dall'applicazione Qt}%
	\label{fig:ui:output}%
\end{figure}

Come per il progetto da riga di comando, questo è un eseguibile che effettua un linking statico verso la libreria, restando dunque un progetto separato da un punto di vista logico.
	\section{Analisi e benchmark}

Per verificare che i metodi si comportassero in maniera corretta, abbiamo provato ad eseguire da interfaccia grafica una serie di test.



%%%%%%%%%%%%%%%%%%%%%%%%%%%%%%%%%%%%%%%%%%%%%%%%%%%%%%%%%%%%%%%%%%%%%%%%%%%%%%%%%%%%%%%%%%%%%%%%%%%%%%%%%%%%%%%%%%%%%%%%%%%%%%%%%%%%%%%%%%%%%%%%%%%%%%%%%%%%%%%%%%%%%%%%%%%%%%%%%%%%%%%%%%%%%%%%%%%%%%%%%%%%%%%%%%%%%%%%%%%%%%%%%%%%%%%%%%%%%%%%%%%%%%%%%%%%%%%%%%%%%%%%%%%%%%%%%%%%%%%

\subsection{Runtime data}

Le seguenti analisi nascono dai dati ricavati dall'esecuzione dei quattro metodi iterativi, i parametri considerati sono: \textit{tolleranza utilizzata} durante la fase di risoluzione, il \textit{tempo di esecuzione espresso in millisecondi}, il \textit{numero di iterazioni} effettuate per arrivare all'output, l'\textit{errore relativo} della soluzione ottenuta in relazione a quella di partenza e il \textit{nome della matrice risolta}. Il fine è quello di trarre delle conclusioni concrete sui quattro metodi iterativi, in particolare sulla loro funzionalità e performance.
(? e trarne una conclusione anche sui migliori parametri da inserire per ogni metodo?).


\subsubsection{Metodi iterativi a confronto}

\paragraph{Tolleranza vs errore relativo}
Abbiamo pensato di mettere in relazione l'errore relativo e la tolleranza utilizzata in fase di risoluzione dei quattro metodi a paragone; l'obiettivo è stabilire in che modo, al variare della tolleranza, il valore dell'errore relativo dei metodi potrebbe differire. In Figura \ref{fig:tolerrmat} e \ref{fig:tolerrmet} sono presenti i grafici relativi alla relazione tra tolleranza ed errore, raggruppati, rispettivamente, per matrici e per metodi.



\begin{figure}%
	\centering
	\subfloat{{\includegraphics[width=0.40\textwidth]{figures/Tolerance vs Relative error/Difference between the 4 methods/spa1.pdf} }}%
	\subfloat{{\includegraphics[width=0.40\textwidth]{figures/Tolerance vs Relative error/Difference between the 4 methods/spa2.pdf} }}%
	\qquad
	\subfloat{{\includegraphics[width=0.40\textwidth]{figures/Tolerance vs Relative error/Difference between the 4 methods/vem1.pdf} }}%
	\subfloat{{\includegraphics[width=0.40\textwidth]{figures/Tolerance vs Relative error/Difference between the 4 methods/vem2.pdf} }}%
	\caption{Grafici tolleranza / errore relativo sulle varie matrici di benchmark}%
	\label{fig:tolerrmat}
\end{figure}


\begin{figure}%
	\centering
	\subfloat{{\includegraphics[width=0.40\textwidth]{figures/Tolerance vs Relative error/Difference between the 4 matrices on the same method/Jacobi.pdf} }}%
	\subfloat{{\includegraphics[width=0.40\textwidth]{figures/Tolerance vs Relative error/Difference between the 4 matrices on the same method/Conjugate Gradient.pdf} }}%
	\qquad
	\subfloat{{\includegraphics[width=0.40\textwidth]{figures/Tolerance vs Relative error/Difference between the 4 matrices on the same method/Gauss-Seidel.pdf} }}%
	\subfloat{{\includegraphics[width=0.40\textwidth]{figures/Tolerance vs Relative error/Difference between the 4 matrices on the same method/Gradient.pdf} }}%
	\caption{Grafici tolleranza / errore relativo rispetto i singoli metodi}%
	\label{fig:tolerrmet}
\end{figure}
%Immagine tolerance-relative_error, spa1, difference between the 4 methods
%Immagine tolerance-relative_error, spa2, difference between the 4 methods
%Immagine tolerance-relative_error, vem1, difference between the 4 methods
%Immagine tolerance-relative_error, vem2, difference between the 4 methods


Osservando i grafici, l'incremento dell'errore relativo è direttamente proporzionale alla crescita della tolleranza utilizzata durante l'esecuzione dei quattro metodi. Il risultato è ragionevole, così come indicato al capitolo \ref{tol/time, diff methods}, all'aumentare della tolleranza il criterio di arresto sarà meno restrittivo, portando ad una soluzione meno accurata, ergo ad un aumento dell'errore relativo.

\paragraph{Tolleranza vs tempo trascorso}\label{tol/time, diff methods}


\begin{figure}%
	\centering
	\subfloat{{\includegraphics[width=0.40\textwidth]{figures/Tolerance vs Elapsed time/Difference between the 4 methods/spa1.pdf} }}%
	\subfloat{{\includegraphics[width=0.40\textwidth]{figures/Tolerance vs Elapsed time/Difference between the 4 methods/spa2.pdf} }}%
	\qquad
	\subfloat{{\includegraphics[width=0.40\textwidth]{figures/Tolerance vs Elapsed time/Difference between the 4 methods/vem1.pdf} }}%
	\subfloat{{\includegraphics[width=0.40\textwidth]{figures/Tolerance vs Elapsed time/Difference between the 4 methods/vem2.pdf} }}%
	\caption{Grafici tolleranza / tempi sulle varie matrici di benchmark}%
	\label{fig:toltimemat}
\end{figure}

\begin{figure}%
	\centering
	\subfloat{{\includegraphics[width=0.40\textwidth]{figures/Tolerance vs Elapsed time/Difference between the 4 matrices on the same method/Jacobi.pdf} }}%
	\subfloat{{\includegraphics[width=0.40\textwidth]{figures/Tolerance vs Elapsed time/Difference between the 4 matrices on the same method/Conjugate Gradient.pdf} }}%
	\qquad
	\subfloat{{\includegraphics[width=0.40\textwidth]{figures/Tolerance vs Elapsed time/Difference between the 4 matrices on the same method/Gauss-Seidel.pdf} }}%
	\subfloat{{\includegraphics[width=0.40\textwidth]{figures/Tolerance vs Elapsed time/Difference between the 4 matrices on the same method/Gradient.pdf} }}%
	\caption{Grafici tolleranza / tempi rispetto i singoli metodi}%
	\label{fig:toltimemet}
\end{figure}
%Immagine tolerance-elapsedTime, spa1, difference between the 4 methods
%Immagine tolerance-elapsedTime, spa2, difference between the 4 methods

Nelle Figure \ref{fig:toltimemat} e \ref{fig:toltimemet} sono rappresentati i grafici relativi all'andamento del tempo rispetto alla tolleranza, raggruppati rispettivamente per matrice e per metodo.
Gauss Seidel è l'algoritmo che per le matrici relativamente dense, surclassa ogni altro metodo. Quando si passa a matrici estremamente sparse, come \path{vem1} e \path{vem2}, l'algoritmo che si comporta meglio sembra essere, invece, il gradiente coniugato.

Dai grafici si evince immediatamente che, all'aumentare della tolleranza, il tempo di esecuzione di ogni algoritmo tende a calare. Questa tendenza è più che ragionevole dato che, al calare della tolleranza, il criterio di arresto diventa più permissivo.
Portando inevitabilmente, per tutti gli algoritmi, ad un numero inferiore di iterazioni, conseguentemente ad un tempo di esecuzione minore ed una soluzione meno precisa.

%%%%%%%%%%%%%%%%%%%%%%%%%%%%%%%%%%%%%%%%%%%%%%%%%%%%%%%%%%%%%%%%%%%%%%%%%%%%%%%%%%%%%%%%%%%%%
%%%%%%%%%%%%%%%%%%%%%%%%%%%%%%%%%%%%%%%%%%%%%%%%%%%%%%%%%%%%%%%%%%%%%%%%%%%%%%%%%%%%%%%%%%%%%
%%%%%%%%%%%%%%%%%%%%%%%%%%%%%%%%%%%%%%%%%%%%%%%%%%%%%%%%%%%%%%%%%%%%%%%%%%%%%%%%%%%%%%%%%%%%%

%Immagine tolerance-elapsedTime, spa1, difference between the 4 matrixes on the same method
%Immagine tolerance-elapsedTime, spa2, difference between the 4 matrixes on the same method
%Immagine tolerance-elapsedTime, vem1, difference between the 4 matrixes on the same method
%Immagine tolerance-elapsedTime, vem2, difference between the 4 matrixes on the same method




	\appendix
	\section{Compilazione ed esecuzione dei progetti}
	Per poter compilare con successo tutti i sottoprogetti, è necessario avere installato CMake alla versione almeno 3.20.
	
	È importante notare che, durante la compilazione de progetti, viene scaricato automaticamente Eigen.
	
	\subsection{CLI}
	È necessario anche avere installato GNU Make.
	Per compilare con successo l'applicazione cli, è necessario eseguire i seguenti comandi:
	\begin{itemize}
		\item Linux/MacOS: 
		\begin{lstlisting}[language=Bash]
mkdir build && cd build
cmake -DCMAKE_BUILD_TYPE=Release ../cli
make \end{lstlisting}
	\end{itemize}
	
	
	Al momento dell'esecuzione del comando CMake, automaticamente, viene importato il progetto della libreria e viene scaricato Eigen, per poter effettuare la compilazione finale. L'output finale generato è l'applicazione chiamata \texttt{iterative-solver} nella cartella \path{build}.
	
\subsection{Qt}
È necessario avere installato prima della compilazione il pacchetto aggiuntivo di Qt \texttt{Charts}.
	\begin{itemize}
	\item Linux/MacOS: 
	\begin{lstlisting}[language=Bash]
mkdir build-qt && cd build-qt
cmake -DCMAKE_BUILD_TYPE=Release -DCMAKE_PREFIX_PATH=<installazione di Qt>/<versione>/<sistema> ../QTInterface
make \end{lstlisting}
\end{itemize}
Il valore del prefix path dipende fortemente dalla propria installazione e dal proprio sistema operativo. In questo caso, viene generata l'applicazione chiamata \texttt{iterative-solver-ui} nella cartella \path{build}.
	\FloatBarrier
	\newpage
	\printbibliography
\end{document}
	