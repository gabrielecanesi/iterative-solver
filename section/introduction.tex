\section{Introduzione}

Risolvere sistemi lineari con matrici di elevate dimensioni può rappresentare un problema per tutti i metodi di risoluzione, in quanto la memoria richiesta potrebbe non essere sufficiente per mantenere tutto il contenuto della matrice.

 Solitamente, però, le matrici di queste dimensioni sono sparse, ovvero contengono poche entrate diverse da 0. Tale informazione consente di definire un insieme di metodi definiti \textbf{iterativi}, i quali basandosi sulla matrice iniziale e sulla soluzione corrente ne generano una successiva. Continuando ad applicarli permettono di ridurre man mano l'errore relativo rispetto alla soluzione corretta.
 
 Si potrebbe pensare di applicare i metodi \textbf{diretti} per generare la soluzione corretta, ma questi sono soggetti al fenomeno del \textbf{fill-in}. La conseguenza di questo metodo è quella di abbassare la convenienza di utilizzo di un formato sparso, portando a un incremento significativo dell'utilizzo della memoria del calcolatore.
 
 Nel seguente progetto viene proposta una libreria, basata su Eigen \cite{Eigen} , il cui scopo è quello di risolvere in maniera iterativa dei sistemi lineari. Attraverso l'interfaccia grafica realizzata si possono osservare i dati in dei grafici, oltre che visualizzare i risultati in una tabella. Nonostante il focus del progetto riguardasse il confronto dei metodi attraverso lo script più "esterno", la libreria vera e propria consente di risolvere qualunque sistema lineare in modo iterativo, fornendo i parametri relativi alla matrice dei coefficienti \textbf{A} e il vettore dei termini noti \textbf{b}. Inoltre si possono specificare alcune informazioni aggiuntive, come il metodo da utilizzare o alcuni parametri specifici per le versioni rilassate dei metodi che li supportano.
 
 
 