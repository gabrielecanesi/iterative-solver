	\appendix
\section{Compilazione ed esecuzione dei progetti}
Per poter compilare con successo tutti i sottoprogetti, è necessario avere installato CMake alla versione almeno 3.20.

È importante notare che, durante la compilazione de progetti, viene scaricato automaticamente Eigen.

\subsection{CLI}
È necessario anche avere installato GNU Make\cite{Make}.
Per compilare con successo l'applicazione cli, è necessario eseguire i seguenti comandi:
\begin{itemize}
	\item Linux/MacOS: 
	\begin{lstlisting}[language=Bash]
mkdir build && cd build
cmake -DCMAKE_BUILD_TYPE=Release ../cli
make \end{lstlisting}
Questi comandi generano l'eseguibile \path{iterative-client}, che può essere eseguito in due modi diversi. Il primo è il comando \path{solve}:
\begin{lstlisting}	
./iterative-client solve <percorso_matrice>
\end{lstlisting}
dopo il percorso della matrice, è possibile aggiungere le opzioni mostrate alla Sezione \ref{sec:cli}; in alternativa, invece, è disponibile il comando \path{test}:
\begin{lstlisting}
./iterative-client test
\end{lstlisting}
Questo si comporta come \path{solve}, ma esegue i metodi su tutte le matrici fornite insieme alla consegna del progetto, che devono trovarsi in \path{../Matrices/}. Anche questo comando accetta le stesse opzioni di \path{solve}.


\item Windows: il modo più veloce per compilare da Windows è quelllo di utilizzare il Windows Subsystem for Linux: in questo modo è possibile compilare esattamente come mostrato in precedenza.
\end{itemize} 


Al momento dell'esecuzione del comando CMake, automaticamente, viene importato il progetto della libreria e viene scaricato Eigen, per poter effettuare la compilazione finale. L'output finale generato è l'applicazione chiamata \texttt{iterative-solver} nella cartella \path{build}.

\subsection{Qt}
È necessario avere installato, prima della compilazione, il pacchetto aggiuntivo di Qt \texttt{Charts}.
\begin{itemize}
	\item Linux/MacOS: 
	\begin{lstlisting}[language=Bash]
mkdir build-qt && cd build-qt
cmake -DCMAKE_BUILD_TYPE=Release -DCMAKE_PREFIX_PATH=<installazione di Qt>/<versione>/<sistema> ../QTInterface
make \end{lstlisting}
In alternativa, è possibile da Qt Creator aprire il file \path{CMakeLists.txt} nella cartella \path{QTInterface} e, in basso a sinistra, cliccare sul bottone di esecuzione dopo avere selezionato come tipo di build "Release" al posto di "Debug".

\item Windows: il modo più veloce per compilare da Windows è quello di aprire sempre Qt Creator ed effettuare la compilazione dall'interfaccia grafica, come spiegato in precedenza.
\end{itemize}
Il valore del prefix path dipende fortemente dalla propria installazione e dal proprio sistema operativo. In questo caso, viene generata l'applicazione chiamata \path{iterative-solver-ui} nella cartella \path{build}.