	\appendix
\section{Compilazione ed esecuzione dei progetti}
Per poter compilare con successo tutti i sottoprogetti, è necessario avere installato CMake alla versione almeno 3.20.

È importante notare che, durante la compilazione de progetti, viene scaricato automaticamente Eigen.

\subsection{CLI}
È necessario anche avere installato GNU Make\cite{Make}.
Per compilare con successo l'applicazione cli, è necessario eseguire i seguenti comandi:
\begin{itemize}
	\item Linux/MacOS: 
	\begin{lstlisting}[language=Bash]
mkdir build && cd build
cmake -DCMAKE_BUILD_TYPE=Release ../cli
make \end{lstlisting}
\end{itemize}


Al momento dell'esecuzione del comando CMake, automaticamente, viene importato il progetto della libreria e viene scaricato Eigen, per poter effettuare la compilazione finale. L'output finale generato è l'applicazione chiamata \texttt{iterative-solver} nella cartella \path{build}.

\subsection{Qt}
È necessario avere installato prima della compilazione il pacchetto aggiuntivo di Qt \texttt{Charts}.
\begin{itemize}
	\item Linux/MacOS: 
	\begin{lstlisting}[language=Bash]
mkdir build-qt && cd build-qt
cmake -DCMAKE_BUILD_TYPE=Release -DCMAKE_PREFIX_PATH=<installazione di Qt>/<versione>/<sistema> ../QTInterface
make \end{lstlisting}
\end{itemize}
Il valore del prefix path dipende fortemente dalla propria installazione e dal proprio sistema operativo. In questo caso, viene generata l'applicazione chiamata \texttt{iterative-solver-ui} nella cartella \path{build}.