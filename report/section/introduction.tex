\section{Introduzione}

La risoluzione di sistemi lineari può diventare rapidamente onerosa, in termini di memoria e operazioni, nel caso in cui le matrici in input siano sparse. Ciò è dovuto al fenomeno del \textbf{fill in}, ovvero la necessità di istanziare matrici dense durante la risoluzione tramite metodi diretti.

I metodi \textbf{iterativi}, mantenendo una rappresentazione sparsa della matrice durante la risoluzione del sistema, rappresentano una soluzione al problema precedentemente illustrato. Tale rappresentazione gli permette di ottimizzare sia i tempi che la memoria necessaria, dato che è necessario iterare e memorizzare solamente i valori che sono effettivamente presenti nella matrice.